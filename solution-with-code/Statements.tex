\baituluan{bai:1}{
\textcolor{UMTRed}{\textbf{(2.00 điểm) }}
%%%%% Nội dung đề bắt đầu ở dòng kế tiếp
Xét đa thức $f(x) = 2x^3 - 3x^2 + 6$.

\begin{enumerate}
    %\begin{multicols}{2} % Nếu muốn các chỉ mục không chia cột thì tắt dòng này đi
        \item[\textcolor{UMTBlue}{\textbf{a.}}] {\textcolor{UMTRed}{
             \textbf{(1.00 điểm) }}
             Hãy tính $f(A)$ với $A = \begin{bmatrix}[]{cccc}
                \phantom{-}1    & \phantom{-}2  & \phantom{-}0  & \phantom{-}1 \\
                -2              & \phantom{-}1  & -1            & \phantom{-}0 \\
                \phantom{-}0    & -2            & \phantom{-}1  & -1 \\
                -1              & \phantom{-}0  & -2            & \phantom{-}1
                \end{bmatrix}.$ 
        }
        
        \item[\textcolor{UMTBlue}{\textbf{b.}}] {\textcolor{UMTRed}{
             \textbf{(1.00 điểm) }}
             Sau khi tính $f(A)$, giả sử $f(A)$ có dạng 
            \(f(A) = \begin{bmatrix}[]{cccc}
                a_{11} & a_{12} & a_{13} & a_{14} \\
                a_{21} & a_{22} & a_{23} & a_{24} \\
                a_{31} & a_{32} & a_{33} & a_{34} \\
                a_{41} & a_{42} & a_{43} & a_{44}
            \end{bmatrix}\). Dùng các phép biến đổi sơ cấp theo dòng đưa ma trận hệ số của hệ phương trình bên dưới về ma trận bậc thang và biện luận số nghiệm của hệ phương trình này
            
            \[
            \begin{cases}
            a_{11}x_1 + a_{12}x_2 + a_{13}x_3 + a_{14}x_4 = 3 \\
            a_{21}x_1 + a_{22}x_2 + a_{23}x_3 + a_{24}x_4 = 1 \\
            a_{31}x_1 + a_{32}x_2 + a_{33}x_3 + a_{34}x_4 = 1 \\
            a_{41}x_1 + a_{42}x_2 + (a_{43} + 4)x_3 + (3m - 4)x_4 = m + 6
            \end{cases}.
            \]
        }

    %\end{multicols} % Nếu muốn các chỉ mục không chia cột thì tắt dòng này đi
\end{enumerate}
} {}



%%%%%%%%%%%%%%%%%%%% Câu 2
\baituluan{bai:2}{
\textcolor{UMTRed}{\textbf{(2.00 điểm) }}
%%%%% Nội dung đề bắt đầu ở dòng kế tiếp
Cho ma trận $A = \begin{bmatrix}[]{ccc}
    -3              & \phantom{-}5  & \phantom{-}1 \\
    \phantom{-}7    & -2            & \phantom{-}4 \\
    \phantom{-}6    & \phantom{-}3  & \phantom{-}9 
\end{bmatrix}.$

\begin{enumerate}
    %\begin{multicols}{2} % Nếu muốn các chỉ mục không chia cột thì tắt dòng này đi
        \item[\textcolor{UMTBlue}{\textbf{a.}}] {\textcolor{UMTRed}{
             \textbf{(1.00 điểm) }}
        Tính định thức của ma trận $A$ và tìm ma trận khả nghịch của $A$ (nếu có).
        }
        
        \item[\textcolor{UMTBlue}{\textbf{b.}}] {\textcolor{UMTRed}{
             \textbf{(1.00 điểm) }}
        Giải hệ phương trình sau bằng cách sử dụng định thức
        $$\begin{cases}
         -3x + 5y + z = 2 \\
         7x - 2y + 4z = 3 \\
         6x + 3y + 9z = 7
        \end{cases}.$$
        } 
    %\end{multicols} % Nếu muốn các chỉ mục không chia cột thì tắt dòng này đi
\end{enumerate}
} {}


%%%%%%%%%%%%%%%%%%%% Câu 3
\baituluan{bai:3}{
\textcolor{UMTRed}{\textbf{(1.50 điểm) }} %%%%% Nội dung đề bắt đầu ở dòng kế tiếp
Xét tập hợp \(S = \{(x,\,y,\,z) \in \mathbb{R}^3 \mid x + 2y - z = 0\}.\)

\begin{enumerate}
    %\begin{multicols}{2} % Nếu muốn các chỉ mục không chia cột thì tắt dòng này đi
        \item[\textcolor{UMTBlue}{\textbf{a.}}] {\textcolor{UMTRed}{
             \textbf{(0.75 điểm) }}
        %%%%% Nội dung tiểu mục bắt đầu ở dòng kế tiếp
        Chứng minh rằng \(S\) là một không gian vector con của \(\mathbb{R}^3\).
        }
        
        \item[\textcolor{UMTBlue}{\textbf{b.}}] {\textcolor{UMTRed}{
             \textbf{(0.75 điểm) }}
        %%%%% Nội dung tiểu mục bắt đầu ở dòng kế tiếp
        Tìm một cơ sở và tính số chiều của không gian \(S\).
        }
        
    %\end{multicols} % Nếu muốn các chỉ mục không chia cột thì tắt dòng này đi
\end{enumerate}
} {}

\baituluan{bai:4}{
\textcolor{UMTRed}{\textbf{(2.00 điểm) }}
Cho các vector $v_1 = (1,\,2,\,3)$, $v_2 = (2,\,4,\,6)$, và $v_3 = (1,\,1,\,0)$ thuộc \(\mathbb{R}^3\). 

\begin{enumerate}
    %\begin{multicols}{2} % Nếu muốn các chỉ mục không chia cột thì tắt dòng này đi
        \item[\textcolor{UMTBlue}{\textbf{a.}}] {\textcolor{UMTRed}{
             \textbf{(1.00 điểm) }}
             Hỏi hệ vector $S = \{v_1,\,v_2,\,v_3\}$ có độc lập tuyến tính không? Nếu không, hãy tìm một cơ sở $B$ của không gian được sinh bởi $S$.
        }
        
        \item[\textcolor{UMTBlue}{\textbf{b.}}] {\textcolor{UMTRed}{
             \textbf{(1.00 điểm) }}
             Giả sử cơ sở tìm được ở trên có dạng $B = \left\{(x_1,\,y_1,\,z_1),\,(x_2,\,y_2,\,z_2),\,\ldots,\,(x_k,\,y_k,\,z_k)\right\}$. Xét cơ sở $B' = \left\{(x_1,\,y_1,\,z_1),\,(2x_2,\,2y_2,\,2z_2),\,\ldots,\,(kx_k,\,ky_k,\,kz_k)\right\}$ của $\text{span}\langle S\rangle$. Biết rằng vector $v = (2,\,3,\,3) \in \text{span}\langle S\rangle$. Hãy tìm tọa độ của $v$ trong các cơ sở $B,\,B'$ và tìm ma trận chuyển cơ sở từ $B$ sang $B'$.
        }
    %\end{multicols} % Nếu muốn các chỉ mục không chia cột thì tắt dòng này đi
\end{enumerate}
}

\baituluan{bai:5}{
\textcolor{UMTRed}{\textbf{(2.00 điểm) }}
%%%%% Nội dung đề bắt đầu ở dòng kế tiếp
Cho ma trận \(A = \begin{bmatrix}[]{cc}
    4 & 1  \\
    0 & 2  \\
\end{bmatrix}.\)

\begin{enumerate}
    %\begin{multicols}{2} % Nếu muốn các chỉ mục không chia cột thì tắt dòng này đi
        \item[\textcolor{UMTBlue}{\textbf{a.}}] {\textcolor{UMTRed}{
             \textbf{(1.00 điểm) }}
             Tính các trị riêng và vector riêng của $A$.
        }
        
        \item[\textcolor{UMTBlue}{\textbf{b.}}] {\textcolor{UMTRed}{
             \textbf{(1.00 điểm) }}
             Ma trận $A$ có chéo hóa được không? Nếu có, hãy tìm ma trận khả nghịch $P$ và ma trận chéo $D$ thỏa mãn  $A = P \cdot D \cdot P^{-1}$. Tính $A^{2024}$.
        }
        
    %\end{multicols} % Nếu muốn các chỉ mục không chia cột thì tắt dòng này đi
\end{enumerate}
} {}


\baituluan{bai:6}{
\textcolor{UMTRed}{\textbf{(0.50 điểm) }}
Cho $A$, $B$ là hai ma trận vuông cấp $n$ thỏa mãn $(A + B)$ khả nghịch. Chứng minh rằng
\[
A \cdot (A + B)^{-1} \cdot B = B \cdot (A + B)^{-1} \cdot A.
\]

} {}