%%%%%%%%%%%%%%%%%%%% Template
\begin{center}
    \textbf{\textcolor{black}{\large CẤU TRÚC DỮ LIỆU VÀ GIẢI THUẬT}}

    DANH SÁCH LIÊN KẾT
\end{center}

% -----------------------------------------------------

% Câu 1
\begin{tcolorbox}[
    colback=cyan!10!white,
    opacityback=0,
    enhanced jigsaw,
    colframe=black!75!white, 
    sharp corners]
    \textcolor{black}{\textbf{Bài 1.}} % Nhập đề bài từ dòng kế tiếp

    \begin{enumerate}
        \item[\textbf{a.}] {
            % Nội dung tiểu mục bắt đầu ở dòng kế tiếp

            % Phải để trống một dòng trước dấu } khi kết thúc
        }
        \item[\textbf{b.}] {
            % Nội dung tiểu mục bắt đầu ở dòng kế tiếp

            % Phải để trống một dòng trước dấu } khi kết thúc
        }
    \end{enumerate}
\end{tcolorbox}

\textcolor{black}{\textbf{Lời giải. }}
\begin{enumerate}
  % \begin{multicols}{2}
      \item[\textcolor{black}{\textbf{a.}}] {
        % Nội dung tiểu mục bắt đầu ở dòng kế tiếp

        % Phải để trống một dòng trước dấu } khi kết thúc
      }
      
      \item[\textcolor{black}{\textbf{b.}}] {
        % Nội dung tiểu mục bắt đầu ở dòng kế tiếp

        % Phải để trống một dòng trước dấu } khi kết thúc
      }
  % \end{multicols}
\end{enumerate}
\textcolor{black}{\textbf{Câu 0. }} 
\begin{itemize}
  % \begin{multicols}{2}
      \item[\textcolor{black}{\textbf{a.}}] {
        %%%%% Nội dung tiểu mục bắt đầu ở dòng kế tiếp
        Quay trở lại bài toán, từ giả thiết suy ra $\det(A) = \pm 1$ và $\det(B) = \pm 1$. Mặt khác $\det(A) \ne \det(B)$ nên $\det(A) + \det(B) = 0$. 
        \hfill \textbf{0.25 điểm}

        Ta có 
        \begin{flalign*}
            &&\det(A + B) \cdot \det(A)  
            &= \det\left((A + B)^T\right) \cdot \det(A) 
            & ~ \\
            && ~ 
            &= \det(A^T + B^T) \cdot \det(A) 
            & \hfill \textbf{0.25 điểm} \\
            && ~ 
            &= \det(A^TA+ B^TA) 
            & ~ \\
            && ~ 
            &= \det(I + B^TA).
            & \textbf{0.25 điểm} 
        \end{flalign*}
        
        Tương tự, ta cũng chứng minh được $\det(A + B) \cdot \det(B) = \det(I + A^TB)$. Mặt khác, $\det(I + B^TA) = \det\left((I + B^TA)^T\right) = \det\left(I + (B^TA)^T\right) = \det\left(I + A^TB\right)$. 
        \hfill \textbf{0.25 điểm}
        
        Do đó, $\det(A + B) \cdot \det(A) = \det(A + B) \cdot \det(B) = -\det(A + B) \cdot \det(A)$ nên $2\cdot \det(A) \cdot \det(A + B) = 0$, kéo theo $\det(A + B) = 0$ (do $\det(A) \ne 0$).  
        \hfill \textbf{0.25 điểm}
        
        %%%%% Phải để trống một dòng trước dấu } khi kết thúc
      }
      
      \item[\textcolor{black}{\textbf{b.}}] {
        %%%%% Nội dung tiểu mục bắt đầu ở dòng kế tiếp
        Chèn hình ảnh vào file tài liệu:
        \begin{center}
            \includegraphics[scale=0.585]{./Resources/Images/image_sample.png}
        \end{center}

        Nếu muốn chèn caption, hãy dùng:
        \begin{figure}[ht]
            \centering
            \includegraphics[scale=0.585]{./Resources/Images/image_sample.png}
            \caption{Chèn hình ảnh mẫu vào tài liệu với caption.}
            \label{fig:image-sample}
        \end{figure}
        
        %%%%% Phải để trống một dòng trước dấu } khi kết thúc
      }
      
      \item[\textcolor{black}{\textbf{c.}}] {
        %%%%% Nội dung tiểu mục bắt đầu ở dòng kế tiếp
        Chèn file code vào file tài liệu:
        
        \lstinputlisting[language=C++, caption=]{./Resources/Files/code_sample.cpp}
        
        %%%%% Phải để trống một dòng trước dấu } khi kết thúc
      }
      
      \item[\textcolor{black}{\textbf{d.}}] {
        %%%%% Nội dung tiểu mục bắt đầu ở dòng kế tiếp
        Chèn bảng vào file tài liệu:
        
        \begin{table}[h]
            \centering
            \begin{tabular}{|C{3cm}|C{3cm}|C{3cm}|}
                \hline
                \bf ABC & \bf XYZ & \bf UVW \\

                \hline
                123 & 123 & 123 \\
                \hline
               123 & 123 & 123 \\
                \hline
            \end{tabular}
            \caption{Chèn bảng mẫu vào tài liệu với caption.}
            \label{tab:table-sample}
        \end{table}
        
        %%%%% Phải để trống một dòng trước dấu } khi kết thúc      
      }
  % \end{multicols}
\end{itemize}

